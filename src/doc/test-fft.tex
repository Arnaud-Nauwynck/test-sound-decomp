\documentclass[english]{article}
\usepackage[T1]{fontenc}
\usepackage[latin9]{inputenc}
\usepackage{babel}
\usepackage[retainorgcmds]{IEEEtrantools}

\begin{document}

\section{General Principles}

Let $S(t)$ be a 1-dimensional time signal (sound) to be compressed.
It is discretized in time by sampling at regular frequency (example 8000 Hz): $t_0, t_1, t_2 \ldots t_N$
$t_n = t_0 + n * \tau$ ,   $\tau=1/F$ 

Each signal measure $S_n=s(t_{n})$ is converted from analogic to digital, so dicretized and encoded to a fixed
integer precision (example: 16 bits).


\subsection{different measures of residu}


\noindent While decomposing a signal $S$ into main components $H_k$: $S=\sum_k H_k + R$ , the residual signal $R = S -
\sum_k H_k$ must be as small as possible: $ \lim_{k \rightarrow \infty} ||R|| = 0 $

\noindent Signal Entropy:
\begin{equation}
E(S) = \sum_n p(S_n) log(S_n)
\end{equation}
where $p(S_n)$ is the probability of obtaining value $S_n$, and $log(S_n)$ is the number of bits (~log in base 2) for
encoding value $S_n$.


\noindent Signal Quadratic Variance:
\begin{equation}
Var(S) = ||S|| = \sum_n S_n^2
\end{equation}
Var(S) represents the norm of the signal = distance to 0. It is linked to scalar product $<f,g> = \sum_n f_n g_n$.


\noindent Signal Absolute Area:
\begin{equation}
AbsArea(S) = \sum_n abs(S_n)
\end{equation}
AbsArea(S) represents the total area of the signal with the x-axis. 

\noindent Signal Maximum Range:
\begin{equation}
MaxRange(S) = \max_n S_n - \min_n S_n
\end{equation}
MaxRange(S) represents the heigth of the horizontal band containing the signal (for a symmetric signal arround 0, it
is the max distance to 0).


\section{Linear Autocorrelation}

$\tilde(X_t)$ linear prediction for $X_t$ knowing $X_{t-1}$: 
\begin{equation}
\tilde(X_t) = A X_{t-1} + B_t, 
X_t= \left( \begin{array}{c} 
x_t \\
x_{t-1} \\
\vdots \\
x_{t-P} 
\end{array}\right), 
A = \left( \begin{array}{ccccc}
  a_1 & a_2 & a_3 & \ldots & a_n\\
  1      & 0   & \ldots \\
  0      & 1   & 0 \\
  \vdots &     & \ddots \\   
\end{array} \right)
\end{equation}

The error between real and predicted value is 
$\tilde(X_t) - X_t = \left( \begin{array}{c} 
x_t - \sum_n a_n x_{t-n} - B_t\\
0\\
0\\
0 \end{array} \right)$ 


The quadratic error is 
\begin{IEEEeqnarray}{rCl}
  Q & = &  Var(\tilde(X) - X) = \sum_n (\tilde(X_n) - X_n)^2 \nonumber\\
  & = & \sum_t (x_t - \sum_n a_n x_{t-n} - B_t) . (x_t - \sum_n a_n x_{t-n} - B_t) \nonumber\\
  & = & \sum_t \left( (x_t-B_t)^2 
  	-2(x_t-B_t).(\sum_n a_n x_{t-n}) 
  	+ (\sum_{n_1} a_{n_1} x_{t-{n_1}}).(\sum_{n_2} a_{n_2} x_{t-{n_2}})
  \right)  \nonumber\\
  & = & \sum_{n_1,n_2} a_{n_1} a_{n_2} \left( \sum_t x_{t-n_1} x_{t-n_2} \right) \nonumber\\
  & & -2 \sum_n a_n \left( \sum_t (x_t-B_t) x_{t-n} \right) \nonumber\\
  & & +  \left( \sum_t (x_t-B_t)^2 \right) \nonumber\\
\end{IEEEeqnarray}

This is a quadratic form of term $a_{ij}$. 
It can be written as 
\begin{equation}
Q(a) = Q^0 - 2 N^t a + a^t M a  
\end{equation}

with $M = \left( \begin{array}{cc} 
& \vdots \\
\ldots & \sum_t x_{t-i} x_{t-j} \\
& \\ 
 \end{array} \right)$
, $N= \left( \begin{array}{c} 
\vdots \\
\sum_t (x_t-B_t) x_{t-i}\\
\vdots \\
\end{array} \right)
$
, $Q^0 = \sum_t (x_t-B_t)^2 $

\noindent A simple linear calculation, analogue to quadratic scalar form $ax^2+bx+c$ \ldots 
gives min for $x=-b/2a$, with value=$c-b^2/2a$. 

\noindent If $M$ is inversible, the minimum of $Q(a)$ is obtained for $a= M^{-1} N$, and has value 
$Q^0 - N^t M^{-1} N$
\noindent If $M$ is not inversible, a similar result can be obtained with the pseudo inverse.

\noindent Note that the result can be interpreted by saying that the variance has reduced by $N^t M^{-1} N$ when
applying the autocorrelation procedure.
  

\section{Least Square Method for Harmonics Amplitude}

The signal is decomposed in $K$ main harmonics, with a residu:
$ S(t_n) = \sum_k S_k(t_n) + R(t)$

We supposed we know frequencies $w_{k}$ and phases $\phi_{k}$ (example: by using zero-crossing algorithms, see next), we
want to adjust optimal amplitude coefficient $c_k$ to miminize quadratic errors.

\noindent For calculation with complex $H_k=c_k \cos(w_{k} t+\phi_{k})$ will be replaced by $H_k=c_k e^{i (w_{k}
t+\phi_{k})}$, and $||x||^2 = x \bar{x}$

\begin{IEEEeqnarray}{rCl}
||R||^2 & = & \sum_n ||S_n - \sum_k H_k(t_n)||^2 \nonumber \\
& = & \sum_n 
 \left( S_n - \sum_{k_1} H_{k_1}(t_n) \right) 
 \left( S_n - \sum_{k_2} \bar{H}_{k_2}(t_n) \right) \nonumber \\
& = & \underbrace{\left( \sum_n S_n^2 \right)}_{c}
 - \left( \sum_k \underbrace{\left( \sum_n S_n (H_{k} + \bar{H}_k)(t_n) \right)}_{b_k} \right)
 + \left( \sum_{k_1,k_2} \underbrace{\sum_n H_{k_1} \bar{H}_{k_2}(t_n)}_{a_{k_1 k_2}} \right) \nonumber
\end{IEEEeqnarray}


\noindent Using $H_k(t_n) = c_k e^{i (w_{k} t+\phi_{k})} $

\noindent it follows $ H_{k} + \bar{H}_k (t) = 2 c_k \cos(w_{k} t+\phi_{k})$ (is real, not complex), and
$$ b_k = 2 c_k \sum_n S_n \cos(w_{k} t+\phi_{k})$$

\noindent Then for $a_{k_1 k_2}$, 
$ H_{k_1} \bar{H}_{k_2}(t) = c_{k_1} c_{k_2} e^{i ((w_{k_1} - w_{k_2}) t + (\phi_{k_1} -
\phi_{k_2}))}$

\noindent summing twice, complex part of $ \ldots(k_1-k_2) + \ldots(k_2-k_1) $ give zero  

$ \frac{1}{2} ( H_{k_1} \bar{H}_{k_2} + H_{k_2} \bar{H}_{k_1}) (t) = \frac{1}{2} c_{k_1} c_{k_2} 2 \cos((w_{k_1} -
w_{k_2}) t + (\phi_{k_1} - \phi_{k_2}))$

\noindent so 
$$ a_{k_1 k_2} = 
c_{k_1} c_{k_2} \sum_n \cos \left((w_{k_1} - w_{k_2}) t_n + \phi_{k_1} - \phi_{k_2} \right)$$
($a_{k_1 k_2}$ is a real symmetric positive matrix)

\noindent Finally,
\begin{equation}
||R||^2 = \underbrace{\left( \sum_n S_n^2 \right)}_{R^0}
 - 2 \sum_k c_k \underbrace{\left( \sum_n S_n \cos(w_{k} t+\phi_{k}) \right)}_{B_k}
 + \sum_{k_1,k_2} c_{k_1} c_{k_2} \underbrace{\left( \sum_n \cos \left((w_{k_1} - w_{k_2}) t_n + \phi_{k_1} -
 \phi_{k_2} \right) \right)}_{A_{k_1 k_2}}
\end{equation}

\noindent This is a quadratic form on vector variable $c_k$ :
\begin{equation}
||R||^2 = R^0 - 2 B^t c + c^t A c  
\end{equation}

\noindent The minimum is reached for $c = A^{-1} B$, and the minimum is $ R^0 - B^t A^{-1} B$

\noindent Note that the result can be interpreted by saying that the variance has reduced by $B^t A^{-1} B$ when
applying the least-square amplitude fitting.



\section{Least Square Method for Harmonics Amplitude Linear Perturbation}

The signal is decomposed in $K$ main harmonics, with a residu:
$ S(t_n) = \sum_k S_k(t_n) + R(t)$

\noindent We have known approximations for frequencies $w_{k}$, phases $\phi_{k}$ and amplitudes $c_k$.

\noindent The $K^{th}$ harmonic
$ S_{k}(t) = c_{k} \cos(w_{k} t+\phi_{k})$  
is replaced by modifying the constant amplitude $c_k$, to obtain a ``non-periodic harmonic'':
$c_k(t) = c_k^0 + c_k^1 (t-t_0) $ 

\noindent This model is usable only for short-time interval!

\noindent The calculation done in previous section is slightly modified.
\begin{IEEEeqnarray}{rCl}
||R||^2 &=& \left( \sum_n S_n^2 \right) \nonumber\\
 & & - 2 \sum_k \left( \sum_n c_k(t_n) S_n \cos(w_{k} t_n+\phi_{k}) \right) \nonumber\\
 & & + \sum_{k_1,k_2} \left( \sum_n c_{k_1}(t_n) c_{k_2}(t_n) \cos \left((w_{k_1} - w_{k_2}) t_n +
 \phi_{k_1} -  \phi_{k_2} \right) \right) \nonumber\\
\end{IEEEeqnarray}

\noindent We want to expand $c_k(t) = c_k^0 + c_k^1 (t-t_0) $, then factorize the variance as a quadratic form on vector
term $c_k^1$

\noindent For ease of calculation, lets note 
$ COS_{kn} = \cos(w_{k} t_n+\phi_{k}) $ 
and 
$ COS\Delta_{k_1 k_2 n} = \cos \left((w_{k_1} - w_{k_2}) t_n + \phi_{k_1} -  \phi_{k_2} \right) \right) $

\begin{IEEEeqnarray}{rCl}
||R||^2 &=& \left( \sum_n S_n^2 \right) \nonumber\\
 & & - 2 \sum_k \sum_n (c_k^0 + c_k^1 (t_n-t_0)) S_n COS_{kn} \right) \nonumber\\
 & & + \sum_{k_1,k_2} \left( \sum_n (c_{k_1}^0 + c_{k_1}^1 (t_n-t_0)) (c_{k_2}^0 + c_{k_2}^1 (t_n-t_0)) 
   COS\Delta_{k_1 k_2 n} \right) \nonumber\\
&=& \sum_n S_n^2  \nonumber\\
 & & - 2 \sum_k c_k^0 \sum_n S_n COS_{kn}
 - 2 \sum_k c_k^1 \left( \sum_n (t_n-t_0) S_n COS_{kn} \right) \nonumber\\
 & & + \sum_{k_1,k_2} 
    \left( \sum_n 
	 	\left( c_{k_1}^0 c_{k_2}^0 
		 	+ (c_{k_1}^0 c_{k_2}^1 + c_{k_1}^1 c_{k_2}^0 ) (t_n-t_0) 
		 	+ c_{k_1}^1 c_{k_2}^1 (t_n-t_0)^2 
	 	\right) 
	   COS\Delta_{k_1 k_2 n} 
   \right) \nonumber\\
&=& \sum_n S_n^2  \nonumber\\
 & & - 2 \sum_k c_k^0 \sum_n S_n COS_{kn}
 - 2 \sum_k c_k^1 \left( \sum_n (t_n-t_0) S_n COS_{kn} \right) \nonumber\\
 & & + \sum_{k_1,k_2} 
    \left( \sum_n 
	 	\left( c_{k_1}^0 c_{k_2}^0 \right) 
	   COS\Delta_{k_1 k_2 n} 
   \right) \nonumber\\
 & & + \sum_{k_1,k_2} 
    \left( \sum_n
    	\underbrace{(c_{k_1}^0 c_{k_2}^1 + c_{k_1}^1 c_{k_2}^0 )}_{= 2 c_{k_1}^1 c_{k_2}^0} (t_n-t_0)
	   COS\Delta_{k_1 k_2 n} 
   \right) \nonumber\\
 & & + \sum_{k_1,k_2} 
    \left( \sum_n 
	 	\left( c_{k_1}^1 c_{k_2}^1 (t_n-t_0)^2 
	 	\right) 
	   COS\Delta_{k_1 k_2 n} 
   \right) \nonumber\\
\end{IEEEeqnarray}

Then\ldots

\begin{IEEEeqnarray}{rCl}
||R||^2 &=& 
 \underbrace{\left( 
 	\sum_n S_n^2
    - 2 \sum_k c_k^0 \sum_n S_n COS_{kn}
    + \sum_{k_1,k_2} c_{k_1}^0 c_{k_2}^0 \sum_n COS\Delta_{k_1 k_2 n} \right)}_{R^0} \nonumber\\
 & & 
 - 2 \sum_k c_k^1 
 	\underbrace{\left( 
 		\sum_n (t_n-t_0) S_n COS_{kn} 
	    - \sum_{k_2} c_{k_2}^0 \sum_n (t_n-t_0) COS\Delta_{k k_2 n}
 	\right)}_{B_k} \nonumber\\
 & & + \sum_{k_1,k_2} 
    c_{k_1}^1 c_{k_2}^1
    \underbrace{\left( \sum_n 
	 	\left( (t_n-t_0)^2 
	 	\right) 
	   COS\Delta_{k_1 k_2 n} 
   \right)}_{A_{k_1 k_2}} \nonumber\\
\end{IEEEeqnarray}

\noindent Finally,
\begin{equation}
||R||^2 = R^0 -2 B^t c^1 + (c^1)^t A (c^1)
\end{equation}

\noindent The minimum is reached for $c^1 = A^{-1} B$, and the minimum is $ R^0 - B^t A^{-1} B$

\noindent Note that the result can be interpreted by saying that the variance has reduced by $B^t A^{-1} B$ when
applying the least-square amplitude linear perturbation procedure.


\section{Least Square Method for Harmonics Frequency Linear Perturbation}

The signal is decomposed in $K$ main harmonics, with a residu:
$ S(t_n) = \sum_k S_k(t_n) + R(t)$

\noindent The $K$^{th} harmonic
$ S_{k}(t) = c_{k} cos(w_{k} t+\phi_{k})$  with $c_k >= 0$, $\phi_k \in [-\pi,\pi]$
is replaced by modifying the constant frequency $w_k$, to obtain a ``non-periodic harmonic'':
$ w_k(t) = w_k^0 + w_k^1 (t-t_0) $ 

\noindent This model is usable only for short-time interval!

\end{document}
